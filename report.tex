% Lab report template, 07/10/2016, v. 2. This template should be used in conjunction with the PDF file generated from it.

% This is the header of your report and will not feature in the report itself. It is useful to write things like the title of the document, the date and what version the document is here. The '%' symbol starts a comment line that will not appear in the final PDF.

% This document was written with TeXstudio and compiled with TeXworks, as TeXstudio cannot compile a pdf itself. Programs like TeXworks or MiKTeX can be used exclusively for writing and compiling a document, but are less user friendly than TeXstudio.

%This template is by no means a guide on how to use LaTeX to write documents, but a brief explanation of the major differences between Microsoft Word and LaTeX is given. LaTeX is an American program, and as such, some of the commands use the American way of spelling i.e. color instead of colour. 

%  USEFUL LINKS FOR LEARNING LaTeX:
%
%   http://en.wikibooks.org/wiki/LaTeX
%
%   http://www.andy-roberts.net/writing/latex/tables
%   http://www.andy-roberts.net/writing/latex/floats_figures_captions
%   http://www.andy-roberts.net/res/writing/latex/symbols.pdf
%

\documentclass[11pt]{article} %This sets the font size and the document class of your report. In this case we use 'article' as that is ideal for shorter reports. 

% LaTeX can be enhanced by the use of packages. These packages can do many things, a few of the most common and useful are used here. They are declared before the document proper, in what is known as the 'preamble'. Packages need to be installed when a .tex file compiles into a .pdf, but should do so automatically.

\usepackage[top=2.54cm, bottom=2.54cm, left=2.75cm, right=2.75cm]{geometry} %This sets the margins of the report.

\usepackage{graphicx} % A package allowing insertion of images into the text.

% Choose your citations style by commenting out one of the following groups. If you decide to change style, you should also delete the .bbl file that you will find in the same folder as your .tex and .pdf files.

% IEEE style citation:
\usepackage{cite}         % A package that creates references in the IEEE style. 
\newcommand{\citet}{\cite} % Use with cite only, so that it understands the natbib-specific \citet command
\bibliographystyle{ieeetr} % IEEE referencing (use in conjunction with the cite package)

%% Author-date style citation:
%\usepackage[round]{natbib} % A package that creates references in the author-date style, with round brackets
%\renewcommand{\cite}{\citep} % For use with natbib only: comment out for the cite package.
%\bibliographystyle{plainnat} % Author-date referencing (use in conjunction with the natbib package)


\usepackage{color} % Allows the colour of the font to be changed by using the '\color' command: This is just to support the blue comments in this template...use standard (black) text in your report.

\linespread{1.2} % Sets the spacing between lines of text.
\setlength{\parindent}{0cm}  % Suppresses indentation of text at the start of a paragraph

\begin{document} % This begins the document proper and ends the pre-amble

\begin{titlepage} % Begins the titlepage of the document
\begin{center} % Starts the beginning of an environment where all text is centered.

{\Huge The Chaotic Pendulum}\\[0.5cm] % [0.5cm] sets the distance between this line and the next.
\textit{Luna Greenberg} and \textit{Hamza Yasin}~\\[0.3cm] % The '\\' starts a new paragraph, and will only work after a paragraph has started, unless we use '~'.
\textit{11017146} and \textit{<Hamza put your student ID here>}~\\[0.3cm]
School of Physics and Astronomy~\\[0.3cm]
University of Manchester~\\[0.3cm]
Second year computational project report~\\[0.3cm]
April 2024~\\[2cm]


\end{center}
{\Large \textbf{Abstract}}~\\[0.3cm]
    <abstract>


\end{titlepage}
\pagenumbering{gobble} % This stops the title page being numbered
\clearpage
\pagenumbering{arabic} % sets the style of page numbering for the report
\setcounter{page}{2} % Starts the numbering at page 2 as typically the first page is not numbered

\newpage % Starts a new page to begin the report on.

\section{Introduction} 
% LaTeX automatically numbers sections and subsections when the command '\section{}' is used. This is useful in very long reports. It does not need a '\\' after it as LaTeX recognises it as a section header.
\label{intro}
% A label allows symbolic cross-referencing using the \ref{} command. 
% \label{} can appear in the text (when they refer to the preceding (sub)section title), equations, tables, or figures. 
% In this case, if you write "Section \ref{intro}" this will be rendered as "Section 1".


\section{Theory}
    Chaos can generally be seen through
    \begin{itemize}
        \item Sensitivity to initial conditions,
        \item Topological mixing, and
        \item Dense periodic orbits.
    \end{itemize}
    There are multiple maps and plots that can be made to examine these prperties. 
    The most common of these is the Poincar\'e section. This is a plot of the phase 
    space of the system, with the position of the pendulum on the x-axis and the velocity 
    on the y-axis. The Poincar\'e section is a useful tool for examining the periodic 
    orbits of the system, and the general method of using the state space of the system 
    is useful in analyzing the other two properties, topological mixing and initial condition
    sensitivity. Other tools include the bifurcation diagram, which shows the states of the
    system as a function of the driving force, as well as the Poincar\'e plot, which is a useful
    tool in examining the underlying structure of the system. \cite{Strogatz2000}\\

\section{Methodology}
    When simulating a chaotic system, it is important to use a numerical method that is very precise and to
    use a small enough time step to ensure that the system is accurately represented. The Euler method is
    a simple method that can be used to simulate the system, but it is not very accurate. The Runge-Kutta
    method is a more accurate method that can be used to simulate the system, and although it may be more
    computationally expensive, the Runge-Kutta method is a good choice for simulating a chaotic system, 
    specifically RK4\cite{Strogatz2000}. In this project, RK4 was used to calculate the steps of angular velocity
    using the equation of motion for the forced, damped pendulum, and Euler's method was used to calculate the
    angle from the angular velocity, as the angular velocities are calculated discretely, there is no way to take
    a half step which is required for RK4. The equation of motion for this system is given by \cite{PhysRevE.53.1579}
    \begin{equation}
        \frac{d^2\theta}{dt^2} = -\frac{g}{R}\sin(\theta) - \frac{b}{M}\frac{d\theta}{dt} + F_d\sin(\Omega_d t)
    \end{equation}
    where $\theta$ is the angle of the pendulum, $t$ is the time since the initial condition, $g$ is the acceleration due to gravity,
    $R$ is the length of the pendulum, $b$ is the damping coefficient, $M$ is the mass of the pendulum, $F_d$ is the driving force,
    and $\Omega_d$ is the frequency of the driving force. This is a second order differential equation, requiring a mesh of 2d initial
    conditions. For this analysis, we assume g, R, b, and M to all be constant, and vary the driving force, $F_d$, and the frequency
    of the driving force, $\Omega_d$. The system will be analyzed using Poincar\'e sections\cite{PoincareSection}, bifurcation
    diagrams \cite{Bifurcation}, Poincar\'e plots\cite{kamen1996poincare}, and the Lyapunov exponent\cite{Lyapunov}, as well as a 
    qualitative analysis of the phase space evolution of the system.\\
    \subsection{Poincar\'e Sections}
        Poincar\'e sections for this project were constructed by analyzing individual initial conditions and plotting the position
        of the pendulum against the velocity of the pendulum until the system reached a periodic orbit. The Poincar\'e section was
        then plotted, and the periodic orbit was analyzed. This was repeated for multiple initial conditions. Determining what makes
        a periodic orbit in a chaotic system with numerical methods is not completely obvious - the error in the numerical method
        was tracked for each initial condition, and the periodic orbit was determined to be when the point returned to within the
        error of the initial condition. The error was determined by the step size of the Runge-Kutta method and Euler method and then
        propagated forward through the equation of motion.\\\vspace{3mm}

        The Poincar\'e section will look different depending on whether or not the system is chaotic. If the system is not chaotic,
        which is theoretically what should be recovered in the case of a small driving force, the Poincar\'e section will look like
        an ellipse or a circle. If the system is chaotic, the Poincar\'e section will look like a dense cloud of points, visually
        not appearing to have any structure. We can perform an ellipse regression test on the poincare map and use the residuals to
        determine whether or not the system is chaotic.\\
    \subsection{Bifurcation Diagrams}
        Bifurcation diagrams were constructed by analyzing the system of the pendulum as a function of the driving force. The system
        was simulated for a range of driving forces, and the system of the pendulum was plotted for many initial conditions over a 
        long time frame. The bifurcation diagram was then analyzed for periodic orbits and chaotic behavior. Bifurcation diagrams were
        created for both the angle of the system and the angular velocity, and were made using a 2d histogram on a log scale of counts\\\vspace{3mm}

        A dense bifurcation diagram is indicative of a chaotic system, while a sparse bifurcation diagram is indicative of a non-chaotic
        system. The system may become chaotic at a certain driving force, and the bifurcation diagram can be used to determine the
        driving force at which it does.\\
    \subsection{Poincar\'e Plots}
        Poincar\'e plots were constructed by plotting the position of the pendulum at a given point on the x axis and the next position of
        the pendulum on the y axis to get a sense of the underlying structure of the system. Structure in the Poincare plot can be indicative
        of certain patterns of behavior and are often used to determine the governing structure of the system. In our case, we can use the
        distribution of points to determine whether or not the system is exhibiting chaotic behavior - if the system is chaotic, the Poincar\'e
        plot will be dense, though there may be a general structure that can give insight into the system.\\
    \subsection{Lyapunov Exponent}
        The Lyapunov exponent is a measure of the sensitivity to initial conditions of a system. It measures the rate at which
        adjacent points in the phase space diverge from each other. A chaotic system will have a positive Lyapunov exponent, while
        a non-chaotic system will not fit the exponential growth of the Lyapunov exponent very well, which can be tested with regression.
    \subsection{Phase Space Analysis}
        The phase space is a 2d plot of the angle of the pendulum on the x-axis and the angular velocity of the pendulum on the y-axis. 
        The phase space is a useful tool for analyzing the behavior of the system, and can be used qualitatively to assess sensitivity 
        to initial conditions and topological mixing. A full GUI was created for analyzing the system qualitatively, however the format
        of the lab report means that only certain snapshots can be shown, and a link to mp4 animations will be provided, as well as a link
        to the project on GitHub, which can be run with python to show the GUI.\\
\section{Results}
    \subsection{Poincar\'e Sections}
    \subsection{Bifurcation Diagrams}
    \subsection{Poincar\'e Plots}
    \subsection{Lyapunov Exponent}
    \subsection{Phase Space Analysis}
\section{Conclusions}
    \subsection{Poincar\'e Sections}
    \subsection{Bifurcation Diagrams}
    \subsection{Poincar\'e Plots}
    \subsection{Lyapunov Exponent}
    \subsection{Phase Space Analysis}
    \subsection{Wholistic Analysis}

\bibliography{report_template_library} % Specifies the bibliography file where our references are stored. If the library file and document are not in the same folder then the file path must also be included.


\end{document}
